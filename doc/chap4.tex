\mychapter{A Review of Alternating Newton Methods}\label{sec:ANT}
\begin{algorithm}[t]
    \caption{Solving problem \eqref{eq:reMF} via alternating minimization.}
    \label{alg:AntFramework}
    \begin{algorithmic}[1]
        \State Given an initial solution $(U, V)$
        \While {stopping condition is not satisfied}
            \State $U   \gets \argmin_{U} f(U,V)$ \label{alg:AntFramework:sub_U}
            \State $V   \gets \argmin_{V} f(U,V)$
        \EndWhile
    \end{algorithmic}
\end{algorithm}

\begin{algorithm}
    \caption{Newton method for solving a sub-problem in step \ref{alg:AntFramework:sub_U} of Algorithm \ref{alg:AntFramework}.}
    \label{alg:LrFrameworkU}
    \begin{algorithmic}[1]
        \State Given $0< \epsilon < 1$, $0<\eta<1$, $\by$, $W$, $H$, and the current $U$, $V$, $f$.
        \State $Q \gets VH^T$ %, $P \gets UW^T$
        \State Compute and cache $\tilde{\by}= (Q \odot UW^T)^T\bsym{1}_{d\times 1}$.
        \State $\bb \gets \tilde{\by} - \by$
        %\State $f \gets \frac{\lambda}{2}(\|U\|_F^2 + \|V\|_F^2) + \frac{1}{2}\sum_{i=1}^l {y_i - \tilde{y}_i}$.
        %\State $f \gets \frac{\lambda}{2}(\|U\|_F^2 + \|V\|_F^2) + \frac{1}{2}\bsym{b}^T \bsym{b}$.
        \For {$k \gets \{0,1,\dots\}$}
            %\State Calculate and store $\tilde{y}_i$ and then obtain $b_i$ by \eqref{eq:LossD}, $\forall i$.
            \State Compute $G \gets \lambda U + Q \bigl(\diag(\bsym{b}) W \bigr)$
            \If {$k=0$}
                \State $\|G^0\|_F \gets \|G\|_F$
            \EndIf
            \If {$\|G\|_F \le \epsilon\|G^0\|_F$}
                \State Output $U$ as the solution of step \ref{alg:AntFramework:sub_U} in Algorithm \ref{alg:AntFramework}.
                \State {\bf break}
            \EndIf
            %\State Compute $D_{ii}$ via \eqref{eq:LossDD}, $\forall i$.
%            \State Run CG in Algorithm \ref{alg:Pcg} to get an update direction $S_u$. \label{alg:LrFrameworkU:CG}
            \State Let $S_u=\bsym{0}_{d\times M}$.
            \State Calculate $R = -G$, $D=R$, and $\gamma^0=\gamma=\|R\|_F^2$.
        \While {$\sqrt{\gamma} > \eta \sqrt{\gamma^0}$}
            %\State $\hat{D}_h \gets \hat{D} \prescript{}{\cdot}{/} \tilde{M}$
            \State $\bz \gets \left( {Q^T}\odot{(WD^T)} \right) \bsym{1}_{d\times 1}$
            \label{alg:Pcg:compz}
            \State $D_h \gets \left(\lambda D + Q \bigl(\diag(\bz) W \bigr)\right)$
            \State $\alpha \gets \gamma / \langle D,D_h \rangle$
            \State $S_u \gets S_u+\alpha D$
            \State $R \gets R-\alpha D_h$
            \State $\gamma^{\text{new}} \gets \|R\|_F^2$
            \State $\beta \gets \gamma^{\text{new}}/\gamma$
            \State $D \gets R+\beta D$
            \State $\gamma \gets \gamma^{\text{new}}$
        \EndWhile
            \State Calculate $\bsym{\Delta} = \left( Q^T \odot (WS_u^T) \right)\bsym{1}_{d\times 1}$ %, \bsym{\tilde{\Delta}}^\prime = \left( (WS_u^T) \odot (HS_v^T) \right)\bsym{1}_{d\times 1}$
%            \State Parpare the following values $\langle{U}{,}{S_u}\rangle,\|S_u\|_F^2,\text{ and } \langle{\tilde{G}}{,}{S_u}\rangle$.
            \State $\delta \gets \frac{\lambda}{2} \left( 2 \langle U, S_u \rangle + \|S_u\|_F^2 \right) + \frac{1}{2} \|\bsym{b}-\theta \Delta \|^2
                - \frac{1}{2}\bsym{b}^T \bsym{b}$
            \State $U \gets U + S_u$
            \State $f \gets f + \delta$
            \State $\tilde{\by} \gets \tilde{\by} + \theta \bsym{\Delta}$
            \State $\bb \gets \tilde{\by} - \by$

            %\State Parpare variables listed in \eqref{eq:LsRequireU} and $\bsym{\tilde{\Delta}} = \left( \pointprod{Q^T}{(WS_u^T)}+\pointprod{P^T}{(HS_v^T)} \right)\bsym{1}_{d\times 1}$
            %\For { $\theta\gets\{1,\beta,\beta^2,\dots\}$ }
            %    %\State $\delta \gets \frac{\lambda'}{2}\left( 2\theta \dotprod{U}{S} + \theta^2\|S\|_F^2 \right) +\sum_{i=1}^l \log\left(1+e^{-y_i(\tilde{y}_i+\theta\tilde{\Delta}_i)}\right)$
            %    %\vspace{-.1cm}
            %    %\Statex \hspace{1.125cm}\phantom{ $\delta \gets$}$- \sum_{i=1}^l \sum_{i=1}^l \log\left(1+e^{-y_i\tilde{y}_i}\right)$.
            %    \State $\delta \gets \frac{\lambda}{2}\left( 2\theta \langle U, S_u \rangle + 2\theta \langle V, S_v \rangle + \theta^2\|S\|_F^2 \right) + \frac{1}{2} \sum_{i=1}^l \xi(\tilde{y}_i+\theta\tilde{\Delta}_i+{\theta}^2\tilde{\Delta}_i^\prime;y_i) - \frac{1}{2}\sum_{i=1}^l \xi(\tilde{y}_i;y_i)$
            %    \If { $ \delta \le \theta \nu \langle G,S \rangle $}
            %        \State $U \gets U +\theta S_u$, $V \gets V +\theta S_v$
            %        \State $f \gets f+ \delta$
            %        \State $\bsym{\tilde{y}} \gets \bsym{\tilde{y}}+\theta\bsym{{\Delta}} +{\theta}^2 \bsym{{\Delta}}^\prime$
            %        \State {\bf break}
            %    \EndIf
            %\EndFor
        \EndFor
    \end{algorithmic}
\end{algorithm}
In \citet{WSC18a}, they solve the optimization problem \eqref{eq:reMF} by a block coordinate descent method. For the two blocks $U$ and $V$, each time one block is fixed while the other is updated. A simple illustration of the procedure is in Algorithm \ref{alg:AntFramework}.
\par For each sub-problem to update one block, \citet{WSC18a} apply a truncated Newton method. Note that because \eqref{eq:min_reMF} is multi-block convex function, each sub-problem is convex. We consider the sub-problem of $U$ as an example to show details of the Newton method. Instead of copying results from \citet{WSC18a}, here we show that materials can be extracted from the more sophisticated algorithm in Section~\ref{sec:NewtonFM} for minimizing both blocks together. To begin, from \eqref{eq:gradUV}-\eqref{eq:Grad}, we have 
\begin{align}
{\nabla}_U {f}(U, V) 
=\lambda U +   Q \bigl(\diag(\bsym{b}) W \bigr)  
\label{eq:AntGrad}.
\end{align}
For the Hessian matrix, from \eqref{eq:H}, 
\begin{align}
\frac{\partial }{\partial \vectorize(U)^T} \frac{\partial f}{\partial \vectorize(U)}
&=\lambda I + \sum_{i=1}^{l} (\bw_i\otimes\bq_i) (\bw_i\otimes\bq_i)^T + \sum_{i=1}^{l} b_i \frac{\partial (\bw_i\otimes\bq_i)}{\partial \vectorize(U)^T}\label{eq:AntH}\\
&=\lambda I + \sum_{i=1}^{l} (\bw_i\otimes\bq_i) (\bw_i\otimes\bq_i)^T\label{eq:AntH_2}.
\end{align}
Note that the last term in \eqref{eq:AntH} is zero because $\bw_i\otimes\bq_i$ is not a function of $U$. Therefore, the Hessian matrix is positive definite. Further, it is a constant matrix independent of $U$. The reason is that when $V$ is fixed, the square loss function considered in \eqref{eq:min_reMF} leads to a convex quadratic function of $U$.

The sub-problem of $U$ becomes equivalent to solving the following linear system by the conjugate gradient method.
\begin{align}
\biggl(\frac{\partial }{\partial \vectorize(U)^T} \frac{\partial f}{\partial \vectorize(U)} \biggr)\bs
&= - \frac{\partial f}{\partial \vectorize(U)}
\label{eq:AntHsG}.
\end{align}
To get the product between the Hessian matrix and a vector 
\begin{equation}
\bs = \vectorize(S_u), 
\label{eq:Ants}
\end{equation}
from (\ref{eq:AntH_2}), (\ref{eq:kronecker_vec}), and (\ref{eq:DefZi}), we first calculate
\begin{equation}
\begin{aligned}
    z_i 
    &=\begin{bmatrix} \bw_i^T\otimes\bq_i^T \end{bmatrix}\vectorize(S_u)\\
    &=\bq_i^T S_u \bw_i , \text{ }i=1,\dots,l
    \label{eq:AntDefZi},
\end{aligned}
\end{equation}
or equivalently
\begin{equation}
    \textbf{\textit{z}} = \biggl({Q^T}\odot{(WS_u^T)}\biggr) \bsym{1}_{d\times 1}
    \label{eq:AntCalcZ}.
\end{equation}
Let $H$ be the Hessian in (\ref{eq:AntH_2}). From (\ref{eq:AntDefZi}), (\ref{eq:AntH_2}), (\ref{eq:Ants}) and  (\ref{eq:kronecker_vec}),
\begin{align}
    H \bs         
                  &= \lambda \vectorize(S_u) + \sum_{i=1}^l z_i (\bw_i\otimes \bq_i) \nonumber \\
                  &= \lambda \vectorize(S_u) +  \sum_{i=1}^l \vectorize\left(z_i \bq_i \bw_i^T\right)  \nonumber  \\
                  &= \lambda \vectorize(S_u) +  \vectorize \Bigl ( Q \bigl(\diag(\textbf{\textit{z}}) W \bigr) \Bigr) \label{eq:AntHv}.
\end{align}
%For line search, now only $U$ is changed. Assume that
%\begin{equation}
%    \tilde{\Delta}_i=(S_u\bw_i)^T(V\bh_i),\; i=1,\dots,l    
%    \label{eq:AntLsRequire}
%\end{equation}
%are available.
%At an arbitrary $\theta$, we can calculate
%\begin{equation}
%    \left((U+\theta S_u)\bw_i\right)^T\left(V\bh_i\right) = \tilde{y}_i + \theta\tilde{\Delta}_i     
%    \label{eq:AntExpTheta}
%\end{equation}
%to get the new output value.
%Now we have 
%\begin{equation*}
%    \begin{aligned}
%        f(U+\theta S_{u},V) - f(U,V)        
%        =& \frac{\lambda}{2}\left( 2\theta \langle{U}{,}{S_u}\rangle + \theta^2\|S_u\|_F^2 \right)\\ 
%        &+ \frac{1}{2} \sum_{i=1}^l (y_i-\tilde{y}_i-\theta\tilde{\Delta}_i)^2 - \frac{1}{2} \sum_{i=1}^l (y_i-\tilde{y}_i)^2        
%    \end{aligned}
%\end{equation*}
%We further maintain
%\begin{equation}    
%    \langle{U}{,}{S_u}\rangle,\|S_u\|_F^2,\text{ and } \tilde{\bg}^T\tilde{\bs}.
%    \label{eq:AntLsRequire1}
%\end{equation}
The computational complexity is
\begin{equation*}
(\text{\# of CG iterations}+2)\times \bbO{ d \times \text{nnz}(W)}.
\end{equation*}
The CG procedure stops if the CG iterate $\bs$ satisfies
\begin{equation}
    \|\ H\bs + \frac{\partial f}{\partial \vectorize(U)} \| \le \eta \| \frac{\nabla f}{\partial \vectorize(U)} \|.
    \label{eq:AntStopMsub}
\end{equation}
Note that because the sub-problem is equivalent to a linear system \eqref{eq:AntHsG}, one single CG procedure is enough to obtain an approximate solution. Thus there is no need to have an outer Newton procedure. The CG procedure is presented in Algorithm \ref{alg:LrFrameworkU}.
Clearly, these procedures are simplified from algorithms considering $U$ and $V$ together. Note that in Algorithm \ref{alg:LrFrameworkU}, we do not need to form $P \gets UW^T$ as in Algorithm \ref{alg:LrFramework} because $UW^T$ is used only once.

To solve the sub-problem over $V$, we apply the same procedure through the following values which are swapped.
$$ U \leftrightarrow V $$
$$ S_u \leftrightarrow S_v $$
$$ \bw_i \leftrightarrow \bh_i $$
$$ \bq_i \leftrightarrow \bp_i $$
$$ W \leftrightarrow H $$
$$ Q \leftrightarrow P $$

Regarding the memory cost, the bottleneck is at similar places to those in Algorithm \ref{alg:LrFramework}. However, the needed memory is only about half of Algorithm \ref{alg:LrFramework}. This can be clearly seen from comparing Line \ref{alg:CG:compz} of Algorithm \ref{alg:CG} and Line \ref{alg:Pcg:compz} of Algorithm \ref{alg:LrFrameworkU}.